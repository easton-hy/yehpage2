%%%%%%%%%%%%%%%%%%%%%%%%%%%%%%%%%%%%%%%%%
% Medium Length Professional CV
% LaTeX Template
% Version 2.0 (8/5/13)
%
% This template has been downloaded from:
% http://www.LaTeXTemplates.com
%
% Original author:
% Trey Hunner (http://www.treyhunner.com/)
%
% Important note:
% This template requires the resume.cls file to be in the same directory as the
% .tex file. The resume.cls file provides the resume style used for structuring the
% document.
%
%%%%%%%%%%%%%%%%%%%%%%%%%%%%%%%%%%%%%%%%%

%----------------------------------------------------------------------------------------
%	PACKAGES AND OTHER DOCUMENT CONFIGURATIONS
%----------------------------------------------------------------------------------------

\documentclass{resume} % Use the custom resume.cls style


\usepackage{enumitem}
\usepackage{gensymb}
\usepackage{hyperref}
\hypersetup{
	colorlinks=true,    
	urlcolor=black,
}
\usepackage[left=0.4 in,top=0.4in,right=0.4 in,bottom=0.4in]{geometry} % Document margins
\newcommand{\tab}[1]{\hspace{.2667\textwidth}\rlap{#1}} 
\newcommand{\itab}[1]{\hspace{0em}\rlap{#1}}
\renewcommand{\baselinestretch}{1}
\name{E\lowercase{aston} Y\lowercase{i} Huang} % Your name
%\address{123 Pleasant Lane \\ City, State 12345} % Your secondary addess (optional)
\address{Personal Website: \url{www.eastonhy.com} \\ Email: eastonhwang@gmail.com }  % Your phone number and email
% In your preamble (before \begin{document})
\usepackage{amsmath}

\begin{document}

\begin{rSection}{Education}
{\bf Ph.D. in Mechanical \& Aerospace Engineering} \\
Rutgers University–New Brunswick, Scbool of Engineering \hfill {\em Aug. 2024 - Now}\\	

\vspace{-1em}

{\bf M.Sc. in Mechanical Engineering} \\
National University of Singapore, College of Design and Engineering \hfill {\em Aug. 2022 - Jan. 2024}\\

\vspace{-1em}

{\bf B.E. in Mechanical Design \& Manufacturing and Their Automation} \\
Dalian University of Technology, School of Mechanical Engineering \hfill {\em Sep. 2018 - Jul. 2021}\\
NUS Suzhou Research Institute
\hfill {\em Sep. 2021 - Jul. 2022}\\
\end{rSection}

\vspace{-1em}


%----------------------------------------------------------------------------------------
%	WORK EXPERIENCE SECTION
%----------------------------------------------------------------------------------------



\begin{rSection}{Research experience}
\begin{rSubsection}{\bf Digital Twins Integrated Finite Element Analysis  }{NUS, Singapore}
	{\bf Instructor: Prof. Andrew Yeh Ching Nee $\&$  Prof. Ong Soh Khim}{2022 - 2024}
	\item {Keywords: digital twins; structural health monitoring; finite element analysis; machine learning; surrogate model.}\\
\end{rSubsection}

\begin{rSubsection}{\bf Intelligent Machine Vision for Surface Condition Inspection  }{NUSRI, Soochow, China}
{\bf Instructor: Prof. Wen Feng Lu  }{2021 - 2022}
	\item {Keywords: machine vision; defect detection; deep learning; denoising.}\\
\end{rSubsection}

\begin{rSubsubsection}{\bf Development of height-adjustable small stool  }{DUT, Dalian, China}
{\bf Undergraduate Innovation and Entrepreneurship Training Program}{2021 - 2022}
{\bf Instructor: Dr. Tieli Zhu}{ }
	\item {Keywords: structural design and optimization.}\\
\end{rSubsubsection}


\begin{rSubsubsection}{\bf Material Damage Modeling based on Multi-sensor Data  }{DUT, Dalian, China}
	{\bf Undergraduate Innovation and Entrepreneurship Training Program at national level}{2020 - 2021}
	{\bf Instructor: Prof. Wei Liu}{ }
	\item {Keywords: drilling of CFRP; multisensor measurement; machine learning.}\\
\end{rSubsubsection}
\end{rSection}

%----------------------------------------------------------------------------------------
\begin{rSection}{Course Projects }
	
\begin{rSubsection}{}{}
{}{}
\item {{\bf Spatio-temporal Prediction based on Data-driven Machine Learning: Earthquakes Case}\\ Module Name: Data-Driven Engineering and Machine Learning}\\

\item {{\bf Predicting Additive Manufacturing Parameters based on Acoustic Analysis}\\ Module Name: Engineering Acoustics}\\

\item {{\bf Calculate and Optimize the Carbon Emissions of Product}\\ Module Name: Sustainable Product Design $\&$ Manufacturing}\\

\item {{\bf Structural Design and Analysis of Quadruped Walking Robot} \\ Module Name: Mechanical Design 1 Course Design}\\

\item {{\bf Gear Reducer Design and Optimization}\\ Module Name: Mechanical Design 2 Course Design}\\
\end{rSubsection}

	%\newpage
	
\end{rSection}
%----------------------------------------------------------------------------------------

%----------------------------------------------------------------------------------------
\begin{rSection}{Other experience}

\begin{rSubsection}{}{}
	{}{}
\item {{\bf China Robotics and Artificial Intelligence Competition}\\ Smart Agriculture Contest, Instructors: Dr. Feilong Wang \& Prof. Shenglan Liu}\\

\item {{\bf Kaggle Competition}\\UW-Madison GI Tract Image Segmentation (UWMGI)}\\

\item {{\bf Dalian University of Technology Varsity Self-Reliance Society}\\Vice President (Junior), Office Manager (Sophomore), Outreach Officer (Freshman)}\\
\end{rSubsection}

\end{rSection}
%----------------------------------------------------------------------------------------
\newpage

\begin{rSection}{SKILLS}
	Unity3D, ANSYS (Workbench $\&$ APDL), Python, MATLAB, Arduino, Solidworks, CAD, Android Studio, HTML.  \\
\end{rSection}

\begin{rSection}{Awards}

\begin{rSubsection}{}{}
{}{}
  \item {First Prize in China Robotics and Artificial Intelligence Competition.}\\
  \item {Learning Excellence Award in DUT.}\\
  \item {Honorary title of Outstanding Officer of Dalian University of Technology Self-Reliance Society.}\\
\end{rSubsection}
\end{rSection}

%----------------------------------------------------------------------------------------
\begin{rSection}{Publications}
\begin{rSubsection}{}{}
{}{}
	\item { {\bf Y. Huang}, A.Y.C. Nee, S.K. Ong. Structural Health Monitoring using Digital Twin Integrated Finite Element Analysis Method. (Being written now) }\\
	\item {{\bf Y. Huang}, 'Intelligent Machine Vision for Detection of Steel Surface Defects with Deep Learning,' 2023 IEEE International Conference on Smart Internet of Things (SmartIoT), Xining, China, 2023, pp. 326-327, doi: 10.1109/SmartIoT58732.2023.00059.}\\
  \item {\fangsong 祝铁丽,时灏烨,林宏彬,李传杰,黄屹,孙先成. 握笔姿势矫正器[P]. 辽宁省:CN214449773U,2021-10-22.}\\
\end{rSubsection}
\end{rSection}


%----------------------------------------------------------------------------------------



\end{document}

